\begin{abstract}
Este trabajo implica el desarrollo de un software capaz de realizar un an\'alisis de cromatograf\'a, puntualmente TLC. TLC (Thin Layer Chromatography) o cromatograf\'ia de capa fina es un t\'ecnica cromatogr\'afica que utiliza una placa vertical que consiste en una fase estacionaria polar adherida a una superficie s\'olida. A partir de esta placa se pueden obtener datos e informaci\'on \'util para el an\'alisis de la composici\'on qu\'imica de las sustancias que se ponen a prueba.

El software desarrollado permite el an\'alisis cromatogr\'afico descripto, al trabajar con la placa mencionada y una foto o im\'agen se podr\'a llevar a cabo la t\'ecnica de cromatograf\'ia y analizar los componentes qu\'imicos. Al final del proceso se entrega un reporte con los resultados completos sobre los compuestos observados. El software permite a la vez el guardado de proyectos para retomar el an\'alisis y una personalizaci\'on de todos los par\'ametros, medidas y \'areas a tener en cuenta. Tambi\'en cuenta con ayudas de gu\'ia constantes en el paso a paso que implementa el software, la posibilidad de insertar comentarios y de retroceder en los pasos ya llevados a cabo. 
\end{abstract}