\begin{abstract}
La Cromatograf\'ia de Capa Fina (o TLC, proveniente de sus siglas en ingl\'es, \textit{Thin Layer Chromatography}) es una t\'ecnica cromatogr\'afica que permite estudiar la composici\'on qu\'imica de las sustancias que se ponen a prueba. B\'asicamente, sobre una placa recubierta de un material absorbente se realiza el sembrado de la sustancia en cuesti\'on y se la ubica verticalmente en una fase estacionaria, que a trav\'es de un proceso de capilaridad, un solvente espec\'ifico descompone la sustancia estudiada en sus diferentes compuestos. El resultado final del proceso de TLC, es una serie de manchas sobre la placa de diferentes tama\~nos y tonalidades, ubicadas a diferentes alturas, dejando en evidencia los diferentes compuestos de la sustancia. Resulta entonces de inter\'es, estudiar, por ejemplo, la cantidad y procentaje de cada compuesto, y a la altura en la que se ubican.

En este trabajo se presenta una herramienta que asiste al ingeniero en el proceso de TLC, mediante el an\'alisis gr\'afico de una fotograf\'ia o imagen de la placa mencionada anteriormente. La herramienta es un asistente que interact\'ua con el ingeniero, permitiendo la personalizaci\'on de los par\'ametros y medidas utilizadas, as\'i como tambi\'en el alamacenamiento y gesti\'on de los experimientos realizados. Finalmente, luego del an\'alisis de TLC, la herramienta genera un reporte detallado sobre los compuestos observados, adem\'as de gr\'aficos comparativos entre las diferentes muestras seleccionadas.
\end{abstract}