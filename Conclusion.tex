\chapter{Conclusi\'on y Trabajos Futuros}
En este trabajo se ha llevado a cabo el desarrollo de un Analizador cromatogr\'afico para muestras de \textit{Biodi\'esel}. Este sistema permite analizar varias muestras procesadas del producto contenidas en una \'unica im\'agen al mismo tiempo, brindando herramientas b\'asicas que permiten recortar y rotar la \'imagen, seleccionar las muestras individualmente de manera sencilla y visualmente amigable, definir los datos individuales de cada muestra (punto de siembra y frente solvente), generar la media muestral y analizar los picos de cada una de ellas obteniendo los datos superficiales absolutos y relativos de cada uno. Como extras el sistema permite llevar la traza del experimento desarollado, guardar y cargar muestras previamente analizadas, exportar los datos de las muestras para ser analizados usando otro \textit{software}, entre otros. El desarrollo de este sistema implic\'o la utilizaci\'on, integraci\'on y aplicaci\'on de diversos conceptos estudiados durante la carrera como patrones arquitecturales y de dise\~no, la utilizaci\'on y manejo de diversas estructuras de datos, el manejo de archivos, an\'alisis matem\'aticos y estad\'isticos, integraci\'on n\'umerica y regla del trapecio, entre otros conceptos. En particular el desarrollo de este sistema implic\'o el an\'alisis de im\'agenes para lo cual se aplicaron diversos conceptos b\'asicos de la computaci\'on gr\'afica y el procesado de im\'agenes.

Si bien, el desarrollo de este trabajo cumpli\'o con las expectativas iniciales, durante su ejecuci\'on se hicieron mas que evidentes ciertas mejoras a incluir en un trabajo futuro. Algunas de estas ser\'ian:
\begin{itemize} \itemsep5pt \parskip0pt \parsep0pt
	\item \textit{Subpantalla de carga:} ciertos procesos del sistema pueden demorar algunos segundos dependiendo del ordenador y del tama\~no de la im\'agen y durante esos segundos no existe indicador que resalte que el software est\'a realizando un proceso, por lo que se puede generar cierta confusi\'on en el usuario. Naturalmente se suele utilizar una pantalla que recubre la vista y se indica que hay una tarea en proceso, una vez finalizada, se vuelve a mostrar la vista.
\end{itemize}
