\chapter{Introducci\'on}
El proceso de Cromatograf\'ia de Capa Fina (TLC) [Sherma y Fried, 2003; Stahl y Ashworth, 1969] puede ser utilizado para analizar la composici\'on qu\'imica de los compuestos que conforman un aceite. B\'asicamente, el proceso consiste en sembrar una gota de aceite sobre una placa de vidrio para TLC con una pel\'icula delgada de silica gel, que mediante el arrastre por capilaridad y a trav\'es de una mezcla de solventes espec\'ifica, separa el aceite en los diferentes compuestos que lo conforman. Este proceso genera una pel\'icula blanca con m\'ultiples manchas coloreadas que se corresponden a los diferentes compuestos que forman el aceite, cuyo tama\~no e intensidad es proporcional a la fracci\'on m\'asica de cada compuesto [Fuchs et al, 2011; Sherma y Fried,  2003]. Si se digitaliza esta placa de TLC, por ejemplo, tomando una fotograf\'ia, es posible mediante una manipulaci\'on gr\'afica de la imagen, cuantificar el porcentaje m\'asico de cada compuesto del aceite.
 
\textit{Christhin} (Chromatography-Riser-Thin) \cite{christin} es un prototipo que ha sido desarrollado por el grupo de Simulaci\'on Aplicada a Procesos Tecnol\'ogicos (SIMAP) de la Facultad de Ingenier\'ia de la Universidad Nacional de R\'io Cuarto. Actualmente, Christhin est\'a implementada como  un paquete para \textit{Octave} \cite{octave}, y permite analizar cuantitativamente los resultados de una placa de TLC de forma r\'apida y eficiente. Sin embargo, Christhin tiene algunas limitaciones, que producen que la utilizaci\'on de la herramienta requiera asistencia constante por parte del usuario: la imagen recibida por la aplicaci\'on debe estar orientada verticalmente, el usuario debe indicar los l\'imites inferior y superior de la placa de TLC digitalizada, el usuario debe identificar manualmente la posici\'on de cada mancha en la placa, etc. Estas actividades no s\'olo disminuyen la usabilidad de la herramienta, sino que adem\'as pueden afectar seriamente la precisi\'on de los resultados obtenidos del an\'alisis. 

En este trabajo proponemos redise\~nar y reimplementar Christhin, teniendo como principales objetivos contribuir a la usabilidad de la herramienta mediante la automatizaci\'on de varias de las tareas que en el prototipo actual deben ser realizadas manualmente, y a la portabilidad, mantenibilidad y extensibilidad de la herramienta mediante un dise\~no adecuado. Para lograr esto, utilizaremos la librer\'ia gr\'afica \textit{ImageJ} \cite{imagej} que permite la manipulaci\'on de las im\'agenes, permiti\'endonos entre otras cosas, detectar autom\'aticamente la posici\'on y contorno de cada una de las manchas formadas por cada compuesto del aceite sobre la placa de TLC, y estudiar alternativas de dise\~nos arquitect\'onico y de clases para conseguir, para esta aplicaci\'on, un buen nivel de flexibilidad en cuanto a mantenibilidad y extensibilidad.

Si bien, en el mercado es posible encontrar f\'acilmente otros sistemas similares que realicen Cromatograf\'ia, muchos de ellos requieren de \textit{Hardware} espec\'ifico para el an\'alisis de las muestras, o son sistemas privativos o de dif\'icil adaptaci\'on en el uso diario. El motivo de desarrollo de este trabajo es implementar un sistema de f\'acil uso que se adapte a las t\'ecnicas \'utilizadas por el equipo de ingenieros liderado por el Dr. Pablo Rossi los cuales procesan y analizan muestras de Biodi\'esel por medio de TLC, obteniendo a trav\'es de la digitalizaci\'on de la placa una imagen de las muestras procesadas que pueden ser f\'acilmente analizadas computacionalmente. 

Para el desarrollo del sistema se utilizar\'a el lenguaje de programaci\'on orientado a objetos \textit{Java} \cite{java} y las libr\'ias base que provee as\'i como tambi\'en la librer\'ia \textit{ImageJ}, para el procesado de las im\'agenes, y la librer\'ia \textit{WebLookAndFeel} \cite{wlaf} para el desarrollo de la interfaz gr\'afica o GUI (por sus siglas en ingles Graphical User Interface).

\section{Cromatograf\'ia}
La ciencia que se encarga de estudiar los diferentes componentes de una sustancia, es la qu\'imica anal\'itica. Una de las t\'ecnicas m\'as usadas para separar los distintos componentes de una mezcla para su posterior estudio, es la cromatograf\'ia. Cuando se deja mover una mezcla sobre un soporte, como la tela o papel (y diversos materiales), los elementos de la mezcla son retenidos en la superficie del soporte de diferente manera y a diferentes velocidades. Al final terminan separ\'andose. Un ejemplo un poco m\'as cotidiano sobre esto, es si se derrama vino sobre un mantel, la mancha que se genera no es uniforme, sino que hay una zona donde predominar\'an tonos azules y otra con la tonalidad roja. Los pigmentos del vino se han separado cromatogr\'aficamente. 

Cromatograf\'ia es la t\'ecnica que permite separar los componentes de una mezcla y su an\'alisis posterior, basado en que las distintas sustancias que forman los componentes de una mezcla se dejan arrastrar a diferentes velocidades sobre un soporte. Este m\'etodo que permite la separaci\'on es f\'isico.

Luego de esto se puede analizar de diferentes formas y se utiliza para lograr la separaci\'on de los componentes de una mezcla como para medir la proporci\'on de cada elemento de la misma.