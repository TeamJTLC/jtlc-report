\chapter{Introducci\'on}
La tecnolog\'ia actual permite que se desarrolle grandes cantidad de programas destinados a diversos avances cient\'ificos, mientras que la ciencia se ha encargado a la vez de poner al desarrollo de software a su servicio. La existencia del software libre ha facilitado la relaci\'on entre la tecnolog\'ia y la ciencia, ya que los programadores cuentan con una diversa cantidad de herramientas para aplicar las t\'ecnicas cient\'ificas. Si bien en muchos casos las t\'ecnicas aplicadas requieren de un hardware especializado para su finalidad, existen t\'ecnicas capaces de aplicarse con recursos m\'as accesibles aunque a veces no se logren resultados con la precisi\'on de t\'encicas con hardware espec\'ifico.

Normalmente la creaci\'on de un programa se lleva a cabo a partir de integraci\'on de diferentes campos de conocimiento. Por ejemplo, si se desarrolla un software que trabaje con planillas de c\'alculo, lo m\'as seguro es que se mezclen \'areas como la computaci\'on (el desarrollador y sus conocimientos) con la econom\'ia, para tomar conocimiento de qu\'e y c\'omo se deber\'ian ejecutar ciertas acciones que este tipo de software requiere; lo mismo sucede en los casos de hacer un software con fines de dise\~no gr\'afico ya que se necesitan los conocimientos de un dise\~nador para poder nutrir de funciones \'utiles para este tipo de profesionales. El programador debe integrar sus conocimientos de alg\'un modo para poder entender qu\'e est\'a haciendo y para qu\'e, logrando as\'i, al nutrirse de informaci\'on, un software que cubra las necesidades del usuario final del software.

En la actualidad existen m\'ultiples recursos tecnol\'ogicos para poder ejecutar an\'alisis con mayor facilidad y precisi\'on. A la vez es posible aplicar algoritmos que ayuden al usuario en el proceso de cualquier an\'alisis. Es posible crear programas que gu\'ien al usuario durante el trabajo y realizar tareas con un alto nivel de detalle y precisi\'on generando informes completos que aporten datos estad\'isticos en muchos estilos y formatos, aportando incluso im\'agenes.

El objetivo primordial de esta tesis es crear un software capaz de realizar el proceso de cromatograf\'ia en capa fina que se realiza en la Universidad Nacional de R\'io Cuarto (UNRC). Dicho proceso se complementa con el software de nombre Christhin. El nuevo software a desarrollar reemplazar\'a a Christhin en el proceso, buscando brindar a los usuarios mayores facultades y beneficios, a fines de mejorar el estudio realizado. Puntualmente se trabaja en biodi\'esel y el an\'alisis de cromatograf\'ia aporta un indicio sobre la pureza y calidad del mismo para pasar luego por un estudio de mayor precisi\'on y an\'alisis que requiere de hardware espec\'ifico, con el que no se cuenta actualmente como un abundante recurso en el pa\'is. Por lo tanto es muy importante determinar mediante TLC cu\'ales son las muestras y componentes que debieran pasar por aquel proceso tan dificultoso. Mejorar la precisi\'on del TLC, es a la vez mejorar las oportunidades de obtener un producto de calidad.

\clearpage

\section{Cromatograf\'ia}

La ciencia que se encarga de estudiar los diferentes componentes de una sustancia, es la qu\'imica anal\'itica. Una de las t\'ecnicas m\'as usadas para separar los distintos componentes de una mezcla para su posterior estudio, es la cromatograf\'ia. Cuando se deja mover una mezcla sobre un soporte, como la tela o papel (y diversos materiales), los elementos de la mezcla son retenidos en la superficie del soporte de diferente manera y a diferentes velocidades. Al final terminan separ\'andose. Un ejemplo un poco m\'as cotidiano sobre esto, es si se derrama vino sobre un mantel, la mancha que se genera no es uniforme, sino que hay una zona donde predominar\'an tonos azules y otra con la tonalidad roja. Los pigmentos del vino se han separado de cromatogr\'aficamente. 

Cromatograf\'ia es la t\'ecnica que permite separar los componentes de una mezcla y su an\'alisis posterior, basado en que las distintas sustancias que forman los componentes de una mezcla se dejan arrastrar a diferentes velocidades sobre un soporte. Este m\'etodo que permite la separaci\'on es f\'isico.

Luego de esto se puede analizar de diferentes formas y se utiliza para lograr la separaci\'on de los componentes de una mezcla como para medir la proporci\'on de cada elemento de la misma.
